\documentclass[12pt]{article}
\begin{document}
\title{mlisp}
\author{Mark Kamradt}
\maketitle
\section*{Built-in Functions}
\subsection*{\texttt{and}}
\texttt{(and \emph{args})} returns \texttt{t} if none of its arguments are \texttt{nil}, otherwise \texttt{nil}.
\texttt{and} evaluates its arguments one at a time and returns \texttt{nil} once an argument is \texttt{nil}.
The remaining arguments are not evaluated.
\begin{center}
\begin{tabular}{rcl}
\texttt{(and t t)} & $\to$ & \texttt{t} \\
\texttt{(and t nil)} & $\to$ & \texttt{nil} \\
\texttt{(and nil t)} & $\to$ & \texttt{nil} \\
\texttt{(and nil nil)} & $\to$ & \texttt{nil} \\
\texttt{(and)} & $\to$ & \texttt{t} \\
\texttt{(and t t t t)} & $\to$ & \texttt{t} \\
\texttt{(and t t nil t)} & $\to$ & \texttt{nil}
\end{tabular}
\end{center}
\subsection*{\texttt{append}}
\texttt{(append x y)} returns the list \texttt{x} with \texttt{y} appended to the end.
\begin{center}
\begin{tabular}{rcl}
\texttt{(append `(a b c) `(x y z))} & $\to$ & \texttt{(a b c x y z)} \\
\texttt{(append `(a b c) nil)} & $\to$ & \texttt{(a b c)} \\
\texttt{(append nil `(a b c))} & $\to$ & \texttt{(a b c)} \\
\texttt{(append `(a b c) `x)} & $\to$ & \texttt{(a b c $.$ x)}
\end{tabular}
\end{center}
\subsection*{\texttt{atom}}
\texttt{(atom x)} returns \texttt{t} if \texttt{x} is an symbol, otherwise \texttt{nil}.
\begin{center}
\begin{tabular}{rcl}
\texttt{(atom `a)} & $\to$ & \texttt{t} \\
\texttt{(atom `(a b c))} & $\to$ & \texttt{nil} \\
\texttt{(atom nil)} & $\to$ & \texttt{nil} \\
\texttt{(atom `nil)} & $\to$ & \texttt{t} \\
\texttt{(atom ())} & $\to$ & \texttt{nil} \\
\texttt{(atom `())} & $\to$ & \texttt{nil}
\end{tabular}
\end{center}
Note that other lisp implementations consider the value \texttt{nil} to be an atom.
mlisp considers \texttt{nil} to be an empty list and, therefore, not an atom.
\subsection*{\texttt{car} and \texttt{cdr}}
\texttt{(car x)} returns the first item in the list \texttt{x}.
\texttt{(cdr x)} returns the remaining items in the list \texttt{x} after the first item is removed.
\begin{center}
\begin{tabular}{rcl}
\texttt{(car `(a b c))} & $\to$ & \texttt{a} \\
\texttt{(car `((a b c) x y)} & $\to$ & \texttt{(a b c)} \\
\texttt{(cdr `(a b c))} & $\to$ & \texttt{(b c)} \\
\texttt{(cdr `((a b c) x y)} & $\to$ & \texttt{(x y)} \\
\texttt{(cdr `(a))} & $\to$ & \texttt{nil}
\end{tabular}
\end{center}
Nested \texttt{car}s and \texttt{cdr}s can be abbreviated. \texttt{(caddr x)} is the same as \texttt{(car (cdr (cdr x)))}.
\subsection*{\texttt{cons}}
\texttt{(cons x y)} returns a list with \texttt{x} as the first item and \texttt{y} as the remaining items.
\begin{center}
\begin{tabular}{rcl}
\texttt{(cons `a `(b c))} & $\to$ & \texttt{(a b c)} \\
\texttt{(cons `a nil)} & $\to$ & \texttt{(a)} \\
\texttt{(cons `a `b)} & $\to$ & \texttt{(a $.$ b)} \\
\texttt{(cons (car `(a b c)) (cdr `(a b c)))} & $\to$ & \texttt{(a b c)}
\end{tabular}
\end{center}
\subsection*{\texttt{copy}}
\texttt{(copy x)} returns a copy of \texttt{x}.
\begin{center}
\begin{tabular}{rcl}
\texttt{(copy `((a b c) x y z)} & $\to$ & \texttt{((a b c) x y z)} \\
\end{tabular}
\end{center}
\subsection*{\texttt{eq}}
\texttt{(eq x y)} returns \texttt{t} if \texttt{x} and \texttt{y} are the same object, otherwise \texttt{nil}.
Identical symbols are considered the same object. Lists that are identical are not the same object except the empty list.
\begin{center}
\begin{tabular}{rcl}
\texttt{(eq `a `a)} & $\to$ & \texttt{t} \\
\texttt{(eq `a `b)} & $\to$ & \texttt{nil} \\
\texttt{(eq `(a b c) `(a b c))} & $\to$ & \texttt{nil} \\
\texttt{(eq () nil)} & $\to$ & \texttt{t}
\end{tabular}
\end{center}
\subsection*{\texttt{equal}}
\texttt{(equal x y)} returns \texttt{t} if \texttt{x} and \texttt{y} are identical.
\begin{center}
\begin{tabular}{rcl}
\texttt{(equal nil nil)} & $\to$ & \texttt{t} \\
\texttt{(equal `a `a)} & $\to$ & \texttt{t} \\
\texttt{(equal `a `b)} & $\to$ & \texttt{nil} \\
\texttt{(equal `(a b c) `(a b c))} & $\to$ & \texttt{t} \\
\texttt{(equal `((x y z) a b c) `((x y z) a b c))} & $\to$ & \texttt{t} \\
\end{tabular}
\end{center}
\subsection*{\texttt{list}}
\texttt{(list \emph{args})} returns its arguments as a single list.
\begin{center}
\begin{tabular}{rcl}
\texttt{(list `a `b `c)} & $\to$ & \texttt{(a b c)} \\
\texttt{(list `(a b) `(c d))} & $\to$ & \texttt{((a b) (c d))} \\
\texttt{(list `a)} & $\to$ & \texttt{(a)} \\
\texttt{(list)} & $\to$ & \texttt{nil} 
\end{tabular}
\end{center}
\subsection*{\texttt{not} and \texttt{null}}
\texttt{(null x)} returns \texttt{t} if \texttt{x} is \texttt{nil}, otherwise \texttt{nil}. 
\texttt{(not x)} is the same as \texttt{(null x)}.
\begin{center}
\begin{tabular}{rcl}
\texttt{(null ())} & $\to$ & \texttt{t} \\
\texttt{(null `())} & $\to$ & \texttt{t} \\
\texttt{(null nil)} & $\to$ & \texttt{t} \\
\texttt{(null `nil)} & $\to$ & \texttt{nil} \\
\texttt{(not `a)} & $\to$ & \texttt{nil} \\
\texttt{(not `(a b c))} & $\to$ & \texttt{nil} \\
\end{tabular}
\end{center}
\subsection*{\texttt{or}}
\texttt{(or \emph{args})} returns \texttt{nil} if all of its arguments are \texttt{nil}, otherwise \texttt{t}.
\texttt{or} evaluates its arguments one at a time and returns \texttt{t} once an argument is not \texttt{nil}.
The remaining arguments are not evaluated.
\begin{center}
\begin{tabular}{rcl}
\texttt{(or t t)} & $\to$ & \texttt{t} \\
\texttt{(or t nil)} & $\to$ & \texttt{t} \\
\texttt{(or nil t)} & $\to$ & \texttt{t} \\
\texttt{(or nil nil)} & $\to$ & \texttt{nil} \\
\texttt{(or)} & $\to$ & \texttt{nil} \\
\texttt{(or t t nil t)} & $\to$ & \texttt{t} \\
\texttt{(or nil nil nil nil)} & $\to$ & \texttt{nil}
\end{tabular}
\end{center}
\end{document}